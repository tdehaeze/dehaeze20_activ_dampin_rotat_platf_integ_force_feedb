% !TeX spellcheck = en_US 
\documentclass{ISMA_USD2020}
%\documentclass{isma2020_Spacing}

%%%%%%%%%%%%
% Packages %
%%%%%%%%%%%%

% The following packages are already loaded in the class definition

	% mfirstuc, inputenc, longtable, tabularx, amsmath, amssymb
	% siunitx, graphicx, epstopdf, pdfpages, bm, relsize,
	% epsfig, epstopdf, eurosym, floatrow, enumitem
	
% If other packages are required for your manuscript below, load them below.



%%%%%%%%%%%%%%%%%%%%%%%%%%%%%%%%%%%%%%%%%%%%%%%%%%%%%%%%%%%%%%%%%%%
% The "hyperref" package may be used in conjunction with PDFLatex %
% to add document information to the generated PDF file. Please,  %
% fill out the same title, author(s) and keywords as on the paper %
% submission form. Note, that the "hypperref" package should be   %
% loaded as the last.                                             %
%                                                                 %
% If you are not using PDFLatex, delete or comment the following  %
% lines. In that case, the conference secretary will add the      %
% document information to your PDF file.                          %
%%%%%%%%%%%%%%%%%%%%%%%%%%%%%%%%%%%%%%%%%%%%%%%%%%%%%%%%%%%%%%%%%%%


\hypersetup{pdftitle  = {Manuscript preparation instructions},
	pdfauthor = {S. Vandemaele, P. Sas},
	pdfkeywords = {ISMA2020, USD2020, template}}

%%%%%%%%%
% Title %
%%%%%%%%%

% Enter below the title of the paper. Use small captions, except for the first letter
\title{Manuscript preparation instructions }

%%%%%%%%%%%%%%%%%%%%%%%%%%%%%%%%
% Author(s) and Affiliation(s) %
%%%%%%%%%%%%%%%%%%%%%%%%%%%%%%%%

% Enter below each author in order of appearance on the paper. 
% Use the format \author[Number of the corresponding affiliation]{Author name (Initial first name. Last name)} 


\author[1,2]{S. Vandemaele}
\author[1,2]{P. Sas}
% Extra authors can be added by copying these lines. 
  
% Enter below each affilliation in order of appearance on the paper. 
% Use the format \affil[Number of the affiliation, used in the author command above]{First line affiliation \NewlineAffil 
% 																					Second line affiliation (\NewAffil if required)}
% To ensure a good spacing on the paper, use the command \NewLineAffil at the end of the first line of the affiliation (=institution name)
% Use the command \NewAffil at the end of the affiliation description if multiple affiliations are added
% Extra affiliations can be added by copying these lines. 		


\affil[1]	{KU Leuven, Department of Mechanical Engineering,\NewLineAffil
			Celestijnenlaan 300, B-3001, Heverlee, Belgium \NewAffil}

\affil[2]	{Member of the ISMA2020 and USD2020 organisation team \NewLineAffil 
			e-mail: \textbf{info@isma-isaac.be} }
		
% Extra affiliations can be added by copying these lines. 


\date{}



\begin{document}

%%%%%%%%%%%%
% Abstract %
%%%%%%%%%%%%

% Enter your abstract in the command below.

\abstract{This document contains the instructions for the preparation of your manuscript for the ISMA2020 and USD2020 conference in Leuven in September 2020. Template files for \LaTeXe and MS Word are available at the conference website (\textbf{http://www.isma-isaac.be}). For any questions about the manuscript preparation instructions, please contact Simon Vandemaele (\textbf{info@isma-isaac.be}).
A digital version of your manuscript should be submitted online through ConfTool \mbox{\textbf{before June 1, 2020}}.}

\maketitle

%%%%%%%%%%%%%%%%%%%%%%%%%%
% Beginning of the paper %
%%%%%%%%%%%%%%%%%%%%%%%%%%


\section{Introduction}

The manuscript will be used as camera-ready artwork and therefore, some basic rules must be followed to ensure quality and standardization of the conference proceedings:
\begin{itemize}
\item be accurate in your typing and thorough in your proof-reading. Your manuscript will be electronically reproduced without any proof-reading or correction,
\item limit the manuscript to \textbf{fifteen (15)} pages, including text, illustrations and references,
\item limit the size of the digital version of your manuscript to \textbf{five megabytes (5 Mb)}.
\end{itemize}
The digital version of your paper will be processed to appear on the on-line conference proceedings. This digital version of your manuscript should be in \textbf{Portable Document Format (PDF)}. Section \ref{s:pdf} discusses various strategies to obtain a PDF file of your manuscript.

\section{Layout of text}

\subsection{General}

The text should fit entirely into a rectangle of \mbox{\textbf{245mm x 166mm}}. Use the following margins on DIN A4-paper \mbox{(297mm x 210mm)}:
\begin{tabbing}
\hspace{4mm} \= $\bullet$ \hspace{1mm} \= top margin \hspace{10mm} \= 30mm,\\
\> $\bullet$ \> bottom margin \> 22mm,\\
\> $\bullet$ \> left margin  \> 22mm,\\
\> $\bullet$ \> right margin \> 22mm.
\end{tabbing}
Use a \textbf{1 column} format. Do \textbf{not} add \textbf{page numbers} nor any other header or footer to the pages.

Use the font Times or Times New Roman 11pt for normal text. The text should be justified. Use single line spacing. Insert a vertical space of 5pt between paragraphs.

Use Helvetica 14pt bold for the first heading, Helvetica 12pt bold for the second heading and Helvetica 11pt bold for the third heading. If Helvetica is not available the Arial font may be used.

Leave one blank line above and one beneath the headings. If a heading falls at the bottom of a page, transfer it to the top of the next page and leave blank space at the bottom.

\subsection{First page}

\subsubsection{Title of the paper}

The title should appear left justified at the top of the first page, in Helvetica 17pt bold. Only capitalize the first letter of the title.

Leave a space of 10mm (28pt) between the title and the authors. Use Times New Roman 11pt \textbf{bold} for the name(s) of the author(s). Type on the next lines (in Times New Roman 11pt) the affiliation of the authors. Use the format as specified in this document: \textless institution/corporation/university\textgreater ~[comma] \textless department/division\textgreater ~[comma] \textless optional lower divisions \textgreater ~[line break] \textless address information (city, state and country)\textgreater. Multiple affiliations should be separated by a blank line. If desired, add \textbf{a single} e-mail address. 

\subsubsection{Abstract}

Include a short abstract (max. 10 lines). Use the word `Abstract' as heading in 14pt Helvetica bold, followed directly by the abstract itself. Leave 2 open lines before starting the text or first heading of the paper.

\subsection{Equations and units}

Equations should be centered and must be allowed sufficient space to ensure clarity. Equations must be numbered consecutively, with the numbers parenthesized at the end of the corresponding line: (1), (2) etc. Example:
\begin{equation}
\sin^2 \phi + \cos^2 \phi = 1 \label{e:circle}
\end{equation}

Use the International System of Units (SI) throughout the paper wherever possible. Acceptable alternates are to use SI units followed by other common units in parentheses, or vice versa, i.e. 25.4mm (1in), 1inch (25.4mm).

\subsection{Figures and tables}

Figures and tables must be numbered consecutively. Each figure and table should be centered, should not exceed the text width and should be accompanied by a small (1 or 2 lines), yet clear, caption. The correct place for a caption is at the bottom of a figure (see Figure~\ref{f:circle}), and at the top of a table (see Table~\ref{t:circle}). The most convenient place for figures is at the top of a page. Leave about 2 lines of space between the actual text and figure (including the caption).

\begin{figure}[t]
\begin{center}
\includegraphics[width=80mm]{circle_rgb}
\caption{$x = \cos \phi$ and $y = \sin \phi$ \label{f:circle}}
\end{center}
\end{figure}

\begin{table}[ht]
\centering
\begin{tabular}{c|cc}
$\phi$   & $x = \cos \phi$ & $y = \sin \phi$ \\
\hline
$0$      &    $1$          &    $0$          \\
$\pi/6$  &  $\sqrt{3}/2$   &    $1/2$        \\
$\pi/3$  &    $1/2$        &  $\sqrt{3}/2$   \\
$\pi/2$  &    $0$          &    $1$
\end{tabular}
\caption{Quarter of a circle \label{t:circle}}
\end{table}

Use only those illustrations pertinent to, and cited in the text. Tables and figures may be \textbf{in color}, but keep in mind that some people will print your manuscript in grayscale for off-line reading. Use preferably 2mm size of lettering and lines of 0.2mm thick on the figures. 


\subsection{Acknowledgements and references}

Acknowledgements, if any, should be typed at the end of the text before the references.

References should be quoted in the text between square brackets (e.g. \cite{heylen1997modal}) and should be listed in order of appearance according to IEEE style, the default style of this document. Examples of how to reference books \cite{heylen1997modal}, journal papers \cite{sas1995active} and publications
in proceedings \cite{boonen2001modified} are shown at the end of these instructions.

\section{Generation of PDF file \label{s:pdf}}

\subsection{\LaTeX ~users}

Pdf\LaTeX{} can be used for the generation of the digital version of your manuscript. This
program can be obtained at e.g. CTAN (the Comprehensive \TeX{} Archive Network, \textbf{http://www.ctan.org}). Pdf\LaTeX{} directly generates a PDF file from a \LaTeX{} file. Furthermore, it allows the use of the `hyperref' package (\textbf{http://www.ctan.org/pkg/hyperref}), which is required to attach some document information to the PDF file (see the \LaTeX{} template file).

\subsection{MS Word users}

MS Word users can either directly generate PDF files or first print to a Postscript file, which will be converted to PDF in a later stage. If the Adobe Acrobat Distiller or the Adobe PDFWriter is installed, then export your Word document in PDF format. Recent versions of MS Word also allow to directly save in PDF format. Otherwise, first install a Postscript printer and use the option `Print to File' to generate a Postscript file. The Postscript file is converted to PDF as described in the next section. In both cases the conference secretary will attach the document information to the manuscript.

\subsection{Postscript to PDF conversion \label{s:ps2pdf}}

The commercially available program Adobe Acrobat Distiller is capable of converting Postscript files to PDF format. Alternative distiller programs, e.g. {\tt ps2pdf} and {\tt ghostview} (\textbf{http://www.cs.wisc.edu/$\sim$ghost/}), exist. Recall that the paper size must be A4 and therefore the online distiller of \textbf{http://www.ps2pdf.com} is not applicable, since it is only capable of exporting PDF files in US letter size. If you do not succeed in generating a proper digital version of your PDF file, please do not hesitate to contact Simon Vandemaele (\textbf{info@isma-isaac.be}).

\subsection{Fonts in PDF \label{s:pdf-fonts}}

In order to circumvent problems concerning the displaying and printing of certain font types (e.g. Asian fonts), it is recommended to use \textbf{only} the \textbf{Times New Roman} and the \textbf{Helvetica} font in the entire document. This includes figures (e.g. generated by {\sc Matlab}), table, equations, captions, \ldots.

%%%%%%%%%%%%%%%%%%%%
% Acknowledgements %
%%%%%%%%%%%%%%%%%%%%

\section*{Acknowledgements}

This PDF file is generated by pdf\LaTeX{} as delivered with the te\TeX{} distribution. The class file `isma2020.cls' and the packages `graphicx' and `hyperref' are explicitly called in the \LaTeX{} file.

%%%%%%%%%%%%%%
% References %
%%%%%%%%%%%%%%

% Include your references below, e.g. by appending a '.bib' file. 
% The style, IEEE, is already pre-defined in the class description.

\bibliography{References}

% If BibTeX is not used, you can enter each bibitem separately using the procedure below

	%\begin{thebibliography}{1}
	%	
	%	\bibitem{heylen}
	%	W. Heylen, S. Lammens, P. Sas,
	%	\textit{Modal Analysis Theory and Testing},
	%	Katholieke Universiteit Leuven, Departement Werktuigkunde, Leuven (1997).
	%	
	%	\bibitem{sas}
	%	P. Sas, C. Bao, F. Augusztinovicz, W. Desmet,
	%	\textit{Active control of sound transmission through a double panel partition},
	%	Journal of Sound and Vibration, Vol. 180, No. 4, Academic Press (1995), pp. 609-625.
	%	
	%	\bibitem{boonen}
	%	R. Boonen, P. Sas,
	%	\textit{Modified Smith Compensation for Feedback Active Noise Control in Ducts},
	%	in \mbox{R. Boone}, editor,
	%	\textit{Proceedings of The 2001 International Congress and Exhibition
	%		on Noise Control Engineering, The Hague, The Netherlands, 2001 August 27-30},
	%	The Hague (2001), pp. 619-624.
	%\end{thebibliography}


%%%%%%%%%%%%%%
% Appendices %
%%%%%%%%%%%%%%

% If none, remove the code below.

\newpage
\appendix

\section{Checklist}

\begin{itemize}
  \item file size:\\
    less than 5 Mb
  \item paper size:\\
    DIN A4-paper \mbox{(297mm x 210mm)}
  \item margins:\\
    top~30mm, bottom~22mm, left~22mm, right~22mm
  \item title:\\
    Helvetica 17pt, bold, left aligned, small captions, only the first letter capital \\
    Correct: \underline{M}anuscript \underline{p}reparation \underline{i}nstructions\\
    Wrong: Manuscript \underline{P}reparation \underline{I}nstructions\\
    Wrong: M\underline{ANUSCRIPT PREPARATION INSTRUCTIONS}
  \item author:\\
    Times New Roman 11pt, bold, left aligned\\
    Format
    \begin{itemize}
    	\item \textless initial(s) first name\textgreater\textless optional: initial(s) middle name\textgreater[space]\textless surname\textgreater
    	\item separate all authors in the list with a [comma].
    \end{itemize}
  \item affiliation:\\
    Times New Roman 11pt, regular, left aligned\\
    Format:\\
    \textless institution/corporation\textgreater[comma]\textless department/division/…\textgreater[comma]\textless optional lab/…\textgreater [enter]\\
    \textless optional address line\textgreater [enter]\\
    \textless optional single email address for the complete paper \textgreater.
  \item abstract\\
    The abstract should not exceed 10 lines, even if the reviewed abstract was longer.
  \item first headings:\\
    Helvetica 14pt, bold, left aligned
  \item second headings:\\
    Helvetica 12pt, bold, left aligned
  \item third headings:\\
    Helvetica 11pt, bold, left aligned
  \item plain text:\\
    Times New Roman 11pt, regular, justified, 5pt whitespace between paragraphs
  \item references:
   	\begin{itemize}
  		\item quotation with brackets [] and chronologically enumerated list with brackets
  		\item IEEE referencing style
	\end{itemize}
  \item equations
  	\begin{itemize}
  		\item font type
  		\item centered and numbered consecutively (1), (2)...
  	\end{itemize}
  \item figures and tables
	\begin{itemize}
		\item font type
		\item centered and numbered consecutively 
		\item clear and concise caption, below each figure and above each table
	\end{itemize}
  \item $\ldots$
\end{itemize}

\newpage

\section{Reference IEEE style}
According to the IEEE reference style, the following information needs to be specified (the name giving corresponds to the BibTeX field names):
\begin{itemize}
  \item book
  	\begin{itemize}
  		\item required: Author, Title, Publisher, Year
  		\item recommended: Address (publisher)
  		\item optional: Edition
  	\end{itemize}
  \item article (journal article)
  	\begin{itemize}
		\item required: Author, Title, Journal (preferred: full name, alternative: official abbreviation), Year
		\item recommended: Volume, Number, Pages
	\end{itemize}  
  \item inproceedings (conference proceedings)
  	\begin{itemize}
		\item required: Author, Title, Booktitle, Year
		\item recommended: Pages
		\item optional: Address
	\end{itemize}
\end{itemize}

\end{document}
